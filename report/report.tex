\documentclass[10pt,fleqn]{article}
\usepackage{microtype}                  % improved microtypography
\usepackage[utf8]{inputenc}             % utf8 encoding
\usepackage{times}                      % times font
\usepackage{mathptmx}                   % math font
\gdef\ttdefault{cmtt}%                                       
\usepackage{graphicx}                   % graphics
\usepackage{xcolor}                     % colors
\usepackage{amsmath}                    % ams math commands
\usepackage[margin=2.54cm]{geometry}    % page layout
\usepackage[english]{babel}             % english typographic rules
\usepackage{titling}                    % to control the position of the title

%----------------------------------------------
% title page
\setlength{\droptitle}{-6em}            % move the title up a bit
\title{ECSE 539 / COMP 533 Assignment 1}
\author{Patrick Diez (260276972), Edward Newell (123456789), Ran Gao (260501253)}

%-------------------------------------------
\begin{document}
\maketitle                              % creates title page


\section{Task Breakdown}
\paragraph{\textbf{Ran Gao}}
	\begin{enumerate}
		\item{plugin map for the withdraw stub.}
		\item{GRL for the customer}
		\item{Overview GRL}
	\end{enumerate}

\paragraph{\textbf{Pat Diez}}
	\begin{enumerate}
		\item{plugin map for the deposit stub.}
		\item{GRL for the Bank}
		\item{Overview GRL}
	\end{enumerate} 

\paragraph{\textbf{Edward Newell}}
	\begin{enumerate}
		\item{Root map, with pseudocode to define the execution path}
		\item{GRL for the Manufacturer}
		\item{Overview GRL}
		\item{Plugin maps for the authentication stub.}
	\end{enumerate}

\section{System Description}
We choose to model an automated teller machine (ATM), which allows banking
customers to make various transactions, including deposits and withdrawals to
their bank accounts.  The ATM is produced by a manufacturer, deployed and 
managed by a bank and used by a customer; we model the goals of these three 
stakeholders and their dependencies.  In a typical scenario, a customer
arives at the ATM, authenticates, and then chooses a kind of transaction.
Depending on the selection of features, the user can authenticate by card and
PIN, or using certain biometrics.  

We modeled two choices of transactions in
detail: deposits and withdrawals.  In general, customer can do multiple 
transactions, which we demonstrate by modelling both a deposit and withdrawal
in our scenario (used for each strategy).  This is a realistic usage scenario,
since, after depositing a paycheque, a customer might well wish to withdraw
cash.  We investigate the performance of the system, in terms of satisfying
stakeholders' goals, by modeling three strategies.  In the first strategy,
we compare our ATM solution to the usual manual solution, wherin the customer
goes to the banking counter and transacts with a (human) teller.  In the 
next two scenarios, we use two different selections of features.  The first
\textit{low security} strategy uses a conventional card/PIN authentication,
and does not include the security camera options.  The second 
\textit{high security} strategy implements biometric authentication and 
includes security cameras.  In terms of its direct impacts, chosing high 
security increases the manufacturing costs, reduces servic calls (by 
prevention of vandalism), increases banking and customer security, and 
increases invasiveness.  Changing from one system configuration to the other
has a complex set of impacts to stakeholder's goals which are especially 
difficult to resolve given the tight interdependencies of the stakeholders.
The modeling tool helped to reason clearly about the selection, and showed that
the high security configuration is better for all stakeholders, despite
having negative impacts on some goals.

\section{Modeling Outcomes}
\paragraph{What went well.}
	The overall architecting of the system, including the division of tasks
	went very smoothly.  The fact that we are using a tool that is
	designed to communicate the structural, behavioral, and intentional 
	aspects of a system made it quite easy to acheive consensus on what we
	were going to model by just diving right in.
\paragraph{What did not go well.}
	One thing that was difficult was understanding how to handle the 
	dependencies between stakeholders.  Of course, the modeling tool has a 
	specific dependency link type designed for that purpose.  While that allows
	us to show that stakeholder's goals are dependant on another, it can 
	arise that one stakeholder $A$ has is its goal ensuring that another 
	stakeholder $B$ achieves her goals.  This makes it tempting to actually
	embed the goal structure of $B$ within $A$'s goals, in order to show that
	$A$ is explicitly concerned about the impact that feature selections has 
	on $B$.  We found that this could be handled by including copies of 
	stakeholder $B$'s high-level goals in stakeholder $A$'s GRL.  Doing so
	means that the impact of feature selections on stakeholder $A$ via effects
	on stakeholder $B$ is properly computed, without having to repeat
	stakeholder $B$'s detailed goal structure in stakeholder $A$'s map.

	With repsect to using the tool itself, some difficulties were encountered.
	The use of global \textsc{id}s for all elements inside the \texttt{.jucm} 
	file makes it 
	likely that concurrent editing of the same file will lead to conflicts
	that are very difficult to resolve.  This is not the same kind of conflict
	which normally arises when concurrent edits to the same file are made in
	a programing language.  It is worth explaining why this is different.  
	First, all of the maps in the \textsc{urn} model are stored in the same 
	file, which
	immediately increases the chance of conflicts.  But what makes this
	particularly problematic is the use of global, incrementing \textsc{id}s 
	to identify model elements.  Unless special steps are taken, 
	concurrent edits are sure to have \textsc{id}s that collide.  When
	editing computer code, such as Java,  programmers are 
	less likely to declare new variables having the same names, and even if
	they do, scoping rules are likely to prevent name-space collisions.  
	We think that, in \texttt{.jucm} files, resolving \textsc{id} collisions 
	requires 
	reverting changes, and traking down and re-labelling all \textsc{id}s 
	involved in 
	one side of the collision which would be very tedious indeed.  We avoided 
	this problem altogether by changing the 
	\texttt{nextGloabalId} attribute in the \texttt{.jucm} file.  This 
	technique does completely avoid collisions, but it is error-prone, 
	because it relies on each developper resetting the \textsc{id} every time they pull
	from the versioning system.

	
\end{document}
