\documentclass[10pt,fleqn]{article}
\usepackage{microtype}                  % improved microtypography
\usepackage[utf8]{inputenc}             % utf8 encoding
\usepackage{times}                      % times font
\usepackage{mathptmx}                   % math font
\gdef\ttdefault{cmtt}%                                       
\usepackage{graphicx}                   % graphics
\usepackage{xcolor}                     % colors
\usepackage{amsmath}                    % ams math commands
\usepackage[margin=2.54cm]{geometry}    % page layout
\usepackage[english]{babel}             % english typographic rules
\usepackage{titling}                    % to control the position of the title

%----------------------------------------------
% title page
\setlength{\droptitle}{-6em}            % move the title up a bit
\title{ECSE 539 / COMP 533 Assignment 1}
\author{Patrick Diez (260276972), Edward Newell (123456789), Ran Gao (260501253)}

%-------------------------------------------
\begin{document}
\maketitle                              % creates title page


\section{Task Breakdown}
\textit{This section should have one paragraph that describes which team member is responsible for which individual parts of the URN model and who contributed to the parts that are a team effort.
Ran Gao is responsible for goal model for customer and scenario model for withdraw plug-in in the root map.}
%
\section{System Description}
\textit{This section should have two paragraphs containing a description of the chosen system.
The system we choose is the automated tellermachine (ATM). ATM is produced by a manufacture, deployed and managed by a bank and used by a customer. Using an ATM, customers can access their bank depostit or credit accounts in order to make various transactions including withdraw, deposit, transfer money.}
%
\section{Modeling Outcomes}
\textit{This section should have four paragraphs discussing what went well and what did not go well when modeling the system, possibly with suggestions on how to improve the modeling experience.}
\end{document}