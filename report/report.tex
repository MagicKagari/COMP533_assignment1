\documentclass[10pt,fleqn]{article}
\usepackage{microtype}                  % improved microtypography
\usepackage[utf8]{inputenc}             % utf8 encoding
\usepackage{times}                      % times font
\usepackage{mathptmx}                   % math font
\gdef\ttdefault{cmtt}%                                       
\usepackage{graphicx}                   % graphics
\usepackage{xcolor}                     % colors
\usepackage{amsmath}                    % ams math commands
\usepackage[margin=2.54cm]{geometry}    % page layout
\usepackage[english]{babel}             % english typographic rules
\usepackage{titling}                    % to control the position of the title

%----------------------------------------------
% title page
\setlength{\droptitle}{-6em}            % move the title up a bit
\title{ECSE 539 / COMP 533 Assignment 1}
\author{Patrick Diez (260276972), Edward Newell (260536886), Ran Gao (260501253)}

%-------------------------------------------
\begin{document}
\maketitle                              % creates title page

\textit{Please see the important note at the end of this document before 
attempting to execute the model.}

\section{Task Breakdown}
Ran Gao implemented the plugin map for the 'Withdraw' stub of the root map,
as well as the goal model for the 'Customer' stakeholder. Patrick Diez
implemented the plugin map for the 'Deposit' stub, as well as the goal model
for the 'Bank' stakeholder. Edward Newell implemented the plugin map for the
'Authentication' stub, as well as the goal model for the 'Manufacturer'
stakeholder. The root map, stakeholders dependencies goal map, and feature
model were contributed to equally by the three team members.

\section{System Description}
We chose to model an automated teller machine (ATM), which allows banking
customers to make various transactions, including deposits and withdrawals to
their bank accounts. The ATM is produced by a manufacturer, deployed and 
managed by a bank, and used by a customer; we model the goals of these three 
stakeholders and their dependencies. In a typical scenario, a customer
arives at the ATM, authenticates, and then chooses a kind of transaction.
Depending on the selection of features, the user can authenticate by card and
PIN, or using certain biometrics.  

We modelled two choices of transactions in
detail: deposits and withdrawals.  In general, a customer can do multiple 
transactions, which we demonstrate by modelling both a deposit and withdrawal
in our scenario (used for each strategy). This is a realistic usage scenario,
since after depositing a cheque a customer might well wish to withdraw
cash. We investigate the performance of the system, in terms of satisfying
stakeholders' goals, by modeling three strategies.  In the first strategy,
we compare our ATM solution to the usual manual solution, wherein the customer
goes to the banking counter and transacts with a (human) teller. In the 
next two strategies, we use two different selections of features. The first
\textit{low security} strategy uses a conventional PIN authentication,
and does not include the security camera options.  The second 
\textit{high security} strategy implements biometric authentication and 
includes security cameras.  In terms of its direct impacts, choosing high 
security increases the manufacturing costs, reduces service calls (by 
prevention of vandalism), increases banking and customer security, and 
increases invasiveness of the customer's privacy.  Changing from one system
configuration to the other
has a complex set of impacts to stakeholders' goals which are especially 
difficult to resolve given the tight interdependencies of the stakeholders.
The modeling tool helped to reason clearly about the selection, and showed that
the high security configuration is better for all stakeholders, despite
having negative impacts on some goals.

\section{Modelling Outcomes}
\paragraph{What went well:}
	The overall architecting of the system, including the division of tasks
	went very smoothly.  The fact that we are using a tool that is
	designed to communicate the structural, behavioral, and intentional 
	aspects of a system made it quite easy to acheive consensus on what we
	were going to model by just diving right in.
\paragraph{What did not go well:}
	One difficulty which arose was the issue of common interests between
	stakeholders in the goal models. As the bank is the manufacturer's client,
	and the customer is the bank's client, it is in the interest of the
	manufacturer and the bank to make their respective products attractive to
	their respective clients. In the first drafts of the model, this resulted
	in duplication (between stakeholders' models) of many elements. Adding to
	the complexity of this issue was the fact that several tasks and/or goals
	common to two or more stakeholders had different contributions to goals
	appearing uniquely in only one of the models. Thus removing the redundant
	elements would make it impossible to represent these contributions, whereas
	retaining them made the goal models unnecessarily large. This issue was
	mitigated by including in each model the tasks and/or goals that contributed
	to elements unique to each model (and thus replicating some of these between
	models), but relegating tasks and/or goals relating to a stakeholder's
	interests to that stakeholder's goal model alone. For example, "Maximize
	Attractiveness to Customers" is a
	softgoal unique to the bank, but the "Minimize Invasiveness" goal (in the
	customer model) contributes to it, and is seen outside the bank actor in
	the bank stakeholder model. Conversely, the "Fingerprint scanner"
	feature/task appears in both the bank and customer models, but its
	contribution to "Minimize Invasiveness" only appears in the customer model
	as "Minimize Invasiveness" is part of the customer actor but not the bank.

	With repsect to using the tool itself, some difficulties were encountered.
	The use of global \textsc{id}s for all elements inside the \texttt{.jucm} 
	file makes it 
	likely that concurrent editing of the same file will lead to conflicts
	that are very difficult to resolve.  This is not the same kind of conflict
	which normally arises when concurrent edits to the same file are made in
	a programming language.  It is worth explaining why this is different.  
	First, all of the maps in the \textsc{urn} model are stored in the same 
	file, which
	immediately increases the chance of conflicts.  But what makes this
	particularly problematic is the use of global, incrementing \textsc{id}s 
	to identify model elements.  Unless special steps are taken, 
	concurrent edits are sure to have \textsc{id}s that collide.  When
	editing computer code, such as Java,  programmers are 
	less likely to declare new variables having the same names, and even if
	they do, scoping rules are likely to prevent namespace collisions.  
	We think that, in \texttt{.jucm} files, resolving \textsc{id} collisions 
	requires 
	reverting changes, and tracking down and re-labelling all \textsc{id}s 
	involved in 
	one side of the collision which would be very tedious indeed.  We avoided 
	this problem altogether by changing the 
	\texttt{nextGloabalId} attribute in the \texttt{.jucm} file.  This 
	technique does completely avoid collisions, but it is error-prone, 
	because it relies on each developper resetting the \textsc{id} every time they pull
	from the versioning system.

	\paragraph{Important Note:} During the course of building this model
	we've run into a constant series of issues with the  software.  On
	several occaisions this interfered with saving, and caused us
	to lose substantial amounts of work.  At some point, it seems that the 
	file became corrupted.  At this time, it is not
	possible to create and execute strategies.  The problem is that, after
	creating a strategy, saving, closing, and re-opening the file, attempting 
	to execute it causes Eclipse to hang.

	We have tried 
	solving this problem several ways, but we cannot figure out what is causing
	it.  After several hours of collective work, our best guess is that 
	there is no issue with the model logic \textit{per se}, but that the file 
	is somehow corrupted.
	It does seem to be related to a bug which occurs when selecting
	certain elements, whereby those elements fail to highlight, though it is 
	difficult to be sure.  In any case,
	it is still possible execute the first strategy
	(``1) Manual Transaction''), and this can be used to manually execute
	the other two strategies as explained below.

	To execute the strategy we call ``ATM - Low security'', stay on strategy
	1, but manually select all features in the feature model except for:
	\begin{itemize}
		\item{Quick withdrawal option,}
		\item{Fingerprint authentication,}
		\item{Camera to see customer's back, and}
		\item{Touchscreen.}
	\end{itemize}

	Now, looking at the scenario, you will see that the scenario now performs
	authentication via the \texttt{Auth Card/Pin} plugin and performs 
	a \texttt{Deposit}, and \texttt{Withdrawal}.

	To execute the third strategy, which we call ``ATM---High security'', 
	while still on the first strategy, ensure that \textit{all} elements are 
	selected
	in the feature model.  You will now see that the scenario executes via
	the \texttt{Auth biometrics} plugin for the \texttt{Authentication} stub,
	while still performing the \texttt{Deposit} and \texttt{Withdrawal}
	transactions.

	We hope you will not mind executing our model manually.  As already
	mentioned, we spent a great deal of time trying to determine the source
	of the problem but to no avail.
\end{document}
